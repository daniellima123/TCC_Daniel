\documentclass[12pt,a4paper]{article}
\usepackage[utf8]{inputenc}
\usepackage[brazilian]{babel}
\usepackage{amsmath}
\usepackage{amsfonts}
\usepackage{amssymb}
\usepackage{enumerate}
\usepackage{graphicx}
\usepackage[left=2cm,right=2cm,top=2cm,bottom=2cm]{geometry}
\usepackage{indentfirst}
\author{Daniel Lima Viegas}

\begin{document}

\section{conceitos básicos}

\subsection{variável resposta e censuras}
Chama-se evento de interesse, aquilo que se deseja encontrar informações sobre a ocorrência. Na análise de sobrevivência, esse evento pode ser a morte de um indivíduo, a cura, um casamento, divórcio ou funcionamento de um dispositivo ou componente de uma máquina. Em análise de sobrevivência, a variável resposta é geralmente o tempo até a ocorrência de um evento de interesse, sendo esse tempo denominado tempo de falha.%citar colosimo;giolo%

A principal característica dos dados de sobrevivência é a presença de censuras, ou seja, observações que por algum motivo não consegue-se determinar o tempo com precisão. Existem três tipos principais de censura, a mais usual é a censura à direita, esta censura acontece quando não se consegue registrar a ocorrência do evento de interesse. Em estudos médicos que analisam o tempo desde a obtenção da doença até a morte do paciente, por exemplo, este tipo de censura acontece quando o paciente é curado, morre por outra razão ou simplesmente não se pode mais observar tal paciente. Este tipo de censura tem três tipos de classificação:

\begin{itemize}
	\item Censura do Tipo I: acontece quando a pesquisa tem um tempo pré-determinado. Ao final do estudo, as observações que não falharam são consideradas censuras. Nesse tipo de estudo, o percentual de censura é descrito como uma variável aleatória.
	
	\item Censura do Tipo II: É encontrada quando se obtém um determinado número de falhas dentro do experimento. Nesse tipo de experimento, o número de falhas deve ser determinado antes de começar o experimento, fazendo com que o número de falhas seja constante. O número de falhas, claramente deve ser menor do que o tamanho da amostra.
	
	\item Censura Aleatória: engloba os outros dois tipos de censura. Acontece quando alguns componentes não podem mais ser acompanhados ou quando o motivo da observação falhar é diferente do que interessa. Esta censura ocorre sem intervenção do pesquisador.

\end{itemize}

Além dessa censura também existem censuras importantes como a censura à esquerda e a censura intervalar. A censura à esquerda ocorre quando o evento ocorre antes do começo do experimento. Por exemplo, deseja-se observar o tempo até uma criança aprender a ler, porém no começo do estudo, algumas crianças podem já ter aprendido a ler sem saber exatamente o tempo em que ela aprendeu, isto é caracterizado como censura à esquerda. % citar colosimo giolo.

A censura intervalar pode ser dita como um caso genérico das outras censuras. Chama-se censura intervalar quando não se sabe o tempo em que ocorreu o evento de interesse
ocorreu, porém sabe-se que ele não ocorreu antes de um determinado tempo. Por exemplo, em um estudo médico é necessário que hajam visitas regulares para a detecção de certas doenças, tal como câncer. Nesse tipo de experimento, sabe-se que a doença apareceu antes do tempo de uma consulta (V), mas também sabe-se que ela apareceu depois de uma consulta (U), ou seja, a doença se manifestou no intervalo [U, V). Quando V = $\infty$
tem-se a censura a direita, e quando a o tempo U = 0, essa censura se torna a esquerda. Daí vem o conhecimento de caso genérico da censura.

\subsection{Tempo}

Como visto anteriormente, a variável resposta do experimento em questão é o tempo até o evento de interesse. Quando esse tempo pode assumir qualquer ponto real não-negativo, descreve-se essa variável como contínua. Esse é o tipo mais comum de variável na análise de sobrevivência, devido a grande diversidade de distribuições contínuas. %% talvez complemente, talvez seja desnecessário.

Em alguns casos, não faz sentido utilizar uma distribuição contínua para descrever o tempo, mas sim uma discreta. Pode-se ter como exemplo o tempo que um aluno leva para sair da universidade, pode levar 8 semestres, 9 semestres e assim por diante, ou seja, nunca vai levar um tempo real e sim um tempo pertencente aos naturais.%% Complementar

\subsection{Funções}

\subsubsection{Função densidade de probabilidade}

Dada uma variável aleatória contínua, não negativa, que represente o tempo de falha de uma observação. Chama-se função densidade, uma função \textit{f}, que descreva a probabilidade de um indivíduo falhar em um intervalo de tempo, quando esse intervalo tende a zero. Sendo assim, essa função descreve a distribuição de probabilidade ao longo do intervalo de zero a infinito.

A partir desta função, é possível obter-se a função de distribuição acumulada, denominada função \textit{F}. No caso contínuo, esta função é obtida a partir do cálculo da integral da função densidade sobre todo seu suporte. Ou seja, dada uma variável aleatória T:

\subsubsection{função densidade de probabilidade}

Segundo Meyer (1983), uma função densidade de probabilidade é uma função que que satisfaz as seguintes condições:%citar Meyer 1983

\begin{enumerate}[i]

	\item $f(x) \ge 0$ para todo $x$,
	\item $\int_{\tiny{-\infty}}^{\tiny{+\infty}} f(x)dx = 1$,
	\item para quaisquer a, b com $-\infty < a < b < +\infty$, teremos $P(a \le X \le b) = \int_a^b f(x)dx$.
\end{enumerate}
	 
		
\subsubsection{função de sobrevivência}

A função de sobrevivência é definida como a probabilidade de um indivíduo não falhar até um determinado tempo t, ou seja, é a probabilidade de uma observação viver além do tempo t. Dada uma variável aleatória T, contínua, não negativa. Pode-se descrever a função de sobrevivência como:

\begin{equation} \label{eq:1}
	\begin{split}
		S(t) & = P(T > t) \\
		& = \int_t^{\infty} f(v)dv 
  	\end{split}
\end{equation} 

\subsubsection{Função de Risco Acumulado}

A função de risco acumulado é uma função que não possui uma interpretação simples, porém possui importância dentro do campo da análise de sobrevivência. Esta função, denotada como $H(t)$, pode ser definida como o logaritmo da função de sobrevivência multiplicado por menos um, ou seja:
\begin{equation} \label{eq:riskcum}
 H(t) = -log(S(t))
\end{equation}

O gráfico desta função pode assumir algumas diferentes formas. Essas formas são utilizadas para determinar possíveis modelos probabilísticos que melhor se adequam aos dados. Com relação ao comportamento da função, o gráfico pode tomar as seguintes formas:

\begin{itemize}
	\item Reta diagonal $\Rightarrow$ Função de risco constante é adequada.
	\item Curva convexa ou côncava $\Rightarrow$ Função
risco é monotonicamente crescente ou decrescente, respectivamente.
	\item Curva convexa e depois côncava $\Rightarrow$ Função risco tem forma de \textbf{U}.
	\item Curva côncava e depois convexa $\Rightarrow$ Função risco tem comportamento unimodal.
\end{itemize}

\subsection{Modelos de Probabilidade}

\subsubsection{Modelo Weibull}
Sendo T uma variável aleatória contínua seguindo uma distribuição Weibull, sua função de densidade é dada por:

$$ f(t) = \dfrac{\gamma}{\alpha^{\gamma}}t^{\gamma-1}\exp-\left\lbrace\left(\dfrac{t}{\alpha}\right)^{\gamma}\right\rbrace, \hspace{1cm} t \ge 0,$$

Onde $\gamma$ e $\alpha$ são constantes positivas e correspondem aos parâmetros de forma e escala, respectivamente. O parâmetro $\alpha$ deve ter a mesma unidade de medida que t, enquanto o parâmetro $\gamma$ não possui nenhuma unidade de medida.

É possível obter a função de sobrevivência para a distribuição Weibull a partir da equação \ref{eq:1}:

\begin{equation} \label{eq:WeiHaz}
S(t) = \exp \left\lbrace - \left( \dfrac{t}{\alpha} \right)^{\gamma} \right\rbrace, \hspace{1cm} t \ge 0
\end{equation}

E através da relação \ref{eq:riskcum}, tem-se que:

\begin{equation} \label{eq:WeRis}
	H(t) = - \left( \dfrac{t}{\alpha} \right)^{\gamma}
\end{equation}

O parâmetro $\gamma$ determina a forma da função de risco. Quando o parâmetro é menor que 1, a função de risco é monótona decrescente, caso o parâmetro maior que 1, a função de risco é monótona crescente e com o parâmetro assumindo valor igual a 1, a variável toma a forma de uma exponencial, que por sua vez, possui função risco constante.

\subsubsection{Modelo Log-Normal}

Considerando-se T uma variável aleatória contínua com distribuição Log-Normal, tem sua função densidade de probabilidade como:

\begin{equation}
 f(t) = \dfrac{1}{(2\pi)^{\frac{1}{2}}\sigma t}\exp\left\lbrace -\dfrac{1}{2} \left( \dfrac{\log(t) - \mu}{\sigma} \right) \right\rbrace, \hspace{1cm} t > 0
\end{equation}

Onde $\mu$ e $\sigma$ são, respectivamente, a média e o desvio padrão da variável aleatória. Ambos os parâmetros são maiores que 0.

As funções de sobrevivência e risco acumulado podem ser descritas envolvendo a função distribuição de probabilidade da Normal Padrão, sendo que esta é:

$$ \Phi(x) =  \int_{-\infty}^{x} \dfrac{1}{\sqrt{2\pi}\sigma} \exp \left\lbrace \dfrac{-u^2}{2\sigma^2} \right\rbrace du $$

A função de sobrevivência da log-normal pode ser facilmente vista como:

\begin{equation} \label{eq:LNSurv}
 S(t) = 1 - \Phi\left(\dfrac{\log(t) - \mu}{\sigma}\right)
\end{equation}

A partir dessa equação, e da relação \ref{eq:riskcum} é simples encontrar a função risco acumulado.
\subsection{Estimadores da função de sobrevivência}

\subsubsection{Estimação simples}

A função de sobrevivência pode ser estimada amostralmente, como a proporção dos dados que não falharam até o tempo t. Esse estimador poderia ser escrito da seguinte forma:

$$ \hat{S}(t) = \dfrac{n^o \ de \ dados \ com \ tempo \ > \ t}{n^o \ total \ de \ individuos}, \forall \ t \ \in t\ge 0$$

Caso os dados sejam ordenados de forma crescente, pode-se representar a função de sobrevivência da seguinte forma:

$$ \hat{S}(t) = \dfrac{n_j - d_j}{n} $$

Onde $n_j$ é o número de indivíduos que podem falhar, $d_j$ é o número de indivíduos que que falharam no tempo e n é o número total de indivíduos.

\subsubsection{Kaplan-Meier}

Os estimadores apresentados acima, não podem ser usados nesse tipo de estudo porque não existe nenhuma forma de se incluir censuras.

O estimador a ser usado nesse trabalho será o estimador não-paramétrico de Kaplan-Meier. Esse estimador é muito popular em pesquisas que usam análise de sobrevivência. O estimador é escrito da seguinte forma:

$$ \hat{S}(t) = \prod_{j:t_{(j)}\le t} \dfrac{n_j - d_j}{n_j}$$

Onde, $n_j$ representa o número de dados em risco de falha, $d_j$ são os dados que falharam no tempo $t_j$, em que, $0 \le t_{(1)} \le \hdots \le t_{(n)}$, são os tempos distintos de falha. Esta técnica não utiliza covariáveis para a estimação, mas pode usar variável categóricas para verificar se as funções estimadas são diferentes. 

A representação gráfica desse método se comporta em uma função da forma de escada, uma vez que a estimação entre o tempo $t_{(j)}$ e $t_{(j+1)}$ é constante.
%% Coloco uma imagem aqui??%%
%% será que eu devo falar sobre consistência e vício do estimador??%%

\end{document}