\documentclass[12pt,a4paper]{article}
\usepackage[utf8]{inputenc}
\usepackage[brazilian]{babel}
\usepackage{amsmath}
\usepackage{amsfonts}
\usepackage{amssymb}
\usepackage{graphicx}
\usepackage[left=2cm,right=2cm,top=2cm,bottom=2cm]{geometry}
\usepackage{indentfirst}
\author{Daniel Lima Viegas}
\begin{document}

\section{conceitos básicos}

\subsection{variável resposta e censuras}
Chama-se evento de interesse, aquilo que se deseja encontrar informações sobre a ocorrência. Na análise de sobrevivência, esse evento pode ser a morte de um indivíduo, a cura, um casamento, divórcio ou funcionamento de um dispositivo ou componente de uma máquina. Em análise de sobrevivência, a variável resposta é geralmente o tempo até a ocorrência de um evento de interesse, sendo esse tempo denominado tempo de falha.%citar colosimo;giolo%

A principal característica dos dados de sobrevivência é a presença de censuras, ou seja, observações que por algum motivo não consegue-se determinar o tempo com precisão. Existem três tipos principais de censura, a mais usual é a censura à direita, esta censura acontece quando não se consegue registrar a ocorrência do evento de interesse. Em estudos médicos que analisam o tempo desde a obtenção da doença até a morte do paciente, por exemplo, este tipo de censura acontece quando o paciente é curado, morre por outra razão ou simplesmente não se pode mais observar tal paciente. Este tipo de censura tem três tipos de classificação:

\begin{itemize}
	\item Censura do Tipo I: acontece quando a pesquisa tem um tempo pré-determinado. Ao final do estudo, as observações que não falharam são consideradas censuras. Nesse tipo de estudo, o percentual de censura é descrito como uma variável aleatória.
	
	\item Censura do Tipo II: É encontrada quando se obtém um determinado número de falhas dentro do experimento. Nesse tipo de experimento, o número de falhas deve ser determinado antes de começar o experimento, fazendo com que o número de falhas seja constante. O número de falhas, claramente deve ser menor do que o tamanho da amostra.
	
	\item Censura Aleatória: engloba os outros dois tipos de censura. Acontece quando alguns componentes não podem mais ser acompanhados ou quando o motivo da observação falhar é diferente do que interessa. Esta censura ocorre sem intervenção do pesquisador.

\end{itemize}

Além dessa censura também existem censuras importantes como a censura à esquerda e a censura intervalar. A censura à esquerda ocorre quando o evento ocorre e não consegue-se determinar o tempo em que ele ocorreu desde o início do experimento. Um exemplo desse tipo de censura seria colocar uma fruta em uma caixa, deixar um determinado e então olhar para verificar se a fruta apodreceu.

A censura intervalar pode ser dita como um caso genérico das outras censuras. Chama-se censura intervalar quando não se sabe o tempo em que ocorreu o evento de interesse
ocorreu, porém sabe-se que ele não ocorreu antes de um determinado tempo. Por exemplo, em um estudo médico é necessário que hajam visitas regulares para a detecção de certas doenças, tal como câncer. Nesse tipo de experimento, sabe-se que a doença apareceu antes do tempo de uma consulta (V), mas também sabe-se que ela apareceu depois de uma consulta (U), ou seja, a doença se manifestou no intervalo [U, V). Quando V = $\infty$
tem-se a censura a direita, e quando a o tempo U = 0, essa censura se torna a esquerda. Daí vem o conhecimento de caso genérico da censura.

\subsection{Tempo}

Como visto anteriormente, a variável resposta do experimento em questão é o tempo até o evento de interesse. Quando esse tempo pode assumir qualquer ponto real não-negativo, descreve-se essa variável como contínua. Esse é o tipo mais comum de variável na análise de sobrevivência, devido a grande diversidade de distribuições contínuas. %% talvez complemente, talvez seja desnecessário.

Em alguns casos, não faz sentido utilizar uma distribuição contínua para descrever o tempo, mas sim uma discreta. Pode-se ter como exemplo o tempo que um aluno leva para sair da universidade, pode levar 8 semestres, 9 semestres e assim por diante, ou seja, nunca vai levar um tempo real e sim um tempo pertencente aos naturais.%% Complementar

\subsection{Funções}

\subsubsection{Função densidade de probabilidade}

Dada uma variável aleatória contínua, não negativa, que represente o tempo de falha de uma observação. Chama-se função densidade, uma função \textit{f}, que descreva a probabilidade de um indivíduo falhar em um intervalo de tempo, quando esse intervalo tende a zero. Sendo assim, essa função descreve a distribuição de probabilidade ao longo do intervalo de zero a infinito.

A partir desta função, é possível obter-se a função de distribuição acumulada, denominada função \textit{F}. No caso contínuo, esta função é obtida a partir do cálculo da integral da função densidade sobre todo seu suporte. Ou seja, dada uma variável aleatória T:

	$$ F(t) = \int_0^{\infty} f(t)dt $$
		
\subsubsection{função de sobrevivência}

A função de sobrevivência é definida como a probabilidade de um indivíduo não falhar até um determinado tempo t, ou seja, é a probabilidade de uma observação viver além do tempo t. Dada uma variável aleatória T, contínua, não negativa. Pode-se descrever a função de sobrevivência como:

\begin{equation*}
	\begin{split}
		S(t) & = P(T > t) \\
		& = \int_t^{\infty} f(v)dv 
  	\end{split}
\end{equation*} 

%% tem que continuar escrevendo sobre as funções%%

\subsection{Estimadores da função de sobrevivência}

\subsubsection{Estimação simples}

A função de sobrevivência pode ser estimada amostralmente, como a proporção dos dados que não falharam até o tempo t. Esse estimador poderia ser escrito da seguinte forma:

$$ \hat{S}(t) = \dfrac{n^o \ de \ dados \ com \ tempo \ > \ t}{n^o \ total \ de \ individuos}, \forall \ t \ \in t\ge 0$$

Caso os dados sejam ordenados de forma crescente, pode-se representar a função de sobrevivência da seguinte forma:

$$ \hat{S}(t) = \dfrac{n_j - d_j}{n} $$

Onde $n_j$ é o número de indivíduos que podem falhar, $d_j$ é o número de indivíduos que que falharam no tempo e n é o número total de indivíduos.

\subsubsection{Kaplan-Meier}

Os estimadores apresentados acima, não podem ser usados nesse tipo de estudo porque não existe nenhuma forma de se incluir censuras.

O estimador a ser usado nesse trabalho será o estimador não-paramétrico de Kaplan-Meier. Esse estimador é muito popular em pesquisas que usam análise de sobrevivência. O estimador é escrito da seguinte forma:

$$ \hat{S}(t) = \prod_{j:t_{(j)}\le t} \dfrac{n_j - d_j}{n_j}$$

Onde, $n_j$ representa o número de dados em risco de falha, $d_j$ são os dados que falharam no tempo $t_j$, em que, $0 \le t_{(1)} \le \hdots \le t_{(n)}$, são os tempos distintos de falha. Esta técnica não utiliza covariáveis para a estimação, mas pode usar variável categóricas para verificar se as funções estimadas são diferentes. 

A representação gráfica desse método se comporta em uma função da forma de escada, uma vez que a estimação entre o tempo $t_{(j)}$ e $t_{(j+1)}$ é constante.
%% Coloco uma imagem aqui??%%
%% será que eu devo falar sobre consistência e vício do estimador??%%

\end{document}