\PassOptionsToPackage{unicode=true}{hyperref} % options for packages loaded elsewhere
\PassOptionsToPackage{hyphens}{url}
%
\documentclass[ignorenonframetext,]{beamer}
\setbeamertemplate{caption}[numbered]
\setbeamertemplate{caption label separator}{: }
\setbeamercolor{caption name}{fg=normal text.fg}
\beamertemplatenavigationsymbolsempty
\usepackage{lmodern}
\usepackage{amssymb,amsmath}
\usepackage{ifxetex,ifluatex}
\usepackage{fixltx2e} % provides \textsubscript
\ifnum 0\ifxetex 1\fi\ifluatex 1\fi=0 % if pdftex
  \usepackage[T1]{fontenc}
  \usepackage[utf8]{inputenc}
  \usepackage{textcomp} % provides euro and other symbols
\else % if luatex or xelatex
  \usepackage{unicode-math}
  \defaultfontfeatures{Ligatures=TeX,Scale=MatchLowercase}
\fi
\usetheme[]{Berkeley}
\usecolortheme{dolphin}
% use upquote if available, for straight quotes in verbatim environments
\IfFileExists{upquote.sty}{\usepackage{upquote}}{}
% use microtype if available
\IfFileExists{microtype.sty}{%
\usepackage[]{microtype}
\UseMicrotypeSet[protrusion]{basicmath} % disable protrusion for tt fonts
}{}
\IfFileExists{parskip.sty}{%
\usepackage{parskip}
}{% else
\setlength{\parindent}{0pt}
\setlength{\parskip}{6pt plus 2pt minus 1pt}
}
\usepackage{hyperref}
\hypersetup{
            pdfborder={0 0 0},
            breaklinks=true}
\urlstyle{same}  % don't use monospace font for urls
\newif\ifbibliography
% Prevent slide breaks in the middle of a paragraph:
\widowpenalties 1 10000
\raggedbottom
\AtBeginPart{
  \let\insertpartnumber\relax
  \let\partname\relax
  \frame{\partpage}
}
\AtBeginSection{
  \ifbibliography
  \else
    \let\insertsectionnumber\relax
    \let\sectionname\relax
    \frame{\sectionpage}
  \fi
}
\AtBeginSubsection{
  \let\insertsubsectionnumber\relax
  \let\subsectionname\relax
  \frame{\subsectionpage}
}
\setlength{\emergencystretch}{3em}  % prevent overfull lines
\providecommand{\tightlist}{%
  \setlength{\itemsep}{0pt}\setlength{\parskip}{0pt}}
\setcounter{secnumdepth}{0}

% set default figure placement to htbp
\makeatletter
\def\fps@figure{htbp}
\makeatother


\title{Análise de Sobrevivência para\\
Dados Grupados}
\author{Daniel Lima Viegas\\
Profª Juliana Betini Fachini Gomes}
\date{24 de novembro de 2017}

\begin{document}
\frame{\titlepage}

\begin{frame}

\logo{\includegraphics[scale=0.14]{unb.jpg}}

\end{frame}

\begin{frame}{%
\protect\hypertarget{introducao}{%
Introdução}}

\begin{itemize}
\item
  Análise de Sobrevivência
\item
  Censura

  \begin{itemize}
  \tightlist
  \item
    Censura a direita
  \item
    Censura a esquerda
  \item
    Censura intervalar
  \end{itemize}
\end{itemize}

\end{frame}

\begin{frame}{%
\protect\hypertarget{justificativa}{%
Justificativa}}

\begin{itemize}
\tightlist
\item
  Grande número de empates nos tempos
\end{itemize}

\begin{center}\includegraphics{pres_tcc1_files/figure-beamer/unnamed-chunk-1-1} \end{center}

\end{frame}

\begin{frame}{%
\protect\hypertarget{objetivo-geral}{%
Objetivo Geral}}

\begin{itemize}
\tightlist
\item
  Propor um modelo de regressão para dados grupados.
\end{itemize}

\end{frame}

\begin{frame}{%
\protect\hypertarget{objetivos-especificos}{%
Objetivos Específicos}}

\begin{itemize}
\tightlist
\item
  Estudar a metodologia de análise de sobrevivência
\item
  Revisar a bibliografia de estudos de sobrevivência com dados grupados
\item
  Estudar as metodologias computacionais presentes no software
  estatístico R
\item
  Estudar o banco de dados para a aplicação de um possível modelo
\item
  Aplicar mátodos para verificar a relação entre a variável tempo e as
  covariáveis no banco
\item
  Propor um modelo de regressão para dados grupados
\end{itemize}

\end{frame}

\end{document}
