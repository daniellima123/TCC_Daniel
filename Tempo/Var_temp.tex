\documentclass[12pt,a4paper]{article}
\usepackage[utf8]{inputenc}
\usepackage[brazil]{babel}
\usepackage{amsmath}
\usepackage{amsfonts}
\usepackage{amssymb}
\usepackage{graphicx}
\usepackage[left=2cm,right=2cm,top=2cm,bottom=2cm]{geometry}
\usepackage{indentfirst}
\author{Daniel Lima Viegas}
\title{Explicando a variável tempo}
\begin{document}

A variável explicada pela análise de sobrevivência é o tempo até o evento de interesse. Quando esse tempo pode assumir qualquer ponto real não-nulo, descreve-se essa variável como contínua. Esse é o tipo mais comum de variável na análise de sobrevivência, devido a grande diversidade de distribuições contínuas. %% talvez complemente, talvez seja desnecessário.

Em alguns casos, não faz sentido utilizar uma distribuição contínua para descrever o tempo, mas sim uma discreta. Pode-se ter como exemplo o tempo que um aluno leva para sair da universidade, pode levar 8 semestres, 9 semestres e assim por diante, ou seja, nunca vai levar um tempo real e sim um tempo inteiro não-nulo.%% Complementar

Um caso mais específico, é o caso onde sabe-se que o evento ocorreu em um intervalo de tempo %% complementar muito

\end{document}